\documentclass[11pt]{article}
    \usepackage{amsmath,amssymb,amsthm}
    \usepackage{graphicx}
    \usepackage[margin=1in]{geometry}
    \usepackage{fancyhdr}
    \usepackage{subcaption} \usepackage{algorithmic}
    \usepackage[makeroom]{cancel}
    \usepackage{algorithm}
    \usepackage{float}
    \usepackage{ragged2e}
    \usepackage{color}
    % \setlength{\parindent}{0pt}
    \setlength\parindent{0pt}
    \setlength{\parskip}{5pt plus 1pt}
    \setlength{\headheight}{13.6pt}       
    \makeatletter
        \newcommand{\pushright}[1]{\ifmeasuring@#1\else\omit\hfill$\displaystyle#1$\fi\ignorespaces}
        \newcommand{\pushleft}[1]{\ifmeasuring@#1\else\omit$\displaystyle#1$\hfill\fi\ignorespaces}
        \makeatother
    
    \pagestyle{fancyplain}
    \lhead{\textbf{\NAME\ (\UIN)}}
    \chead{\textbf{HW\HWNUM}}
    \rhead{ECE 452, Robotics: Algorithm/Control}
    \sloppy
    \definecolor{lightgray}{gray}{0.5}
    \begin{document}\raggedright
    %Section A==============Change the values below to match your information==================
    \newcommand\NAME{Amrutha Varshini Ramesh}  % your name
    \newcommand\UIN{663365516}     % your andrew id
    \newcommand\HWNUM{2}              % the homework number
    %Title and author
    % \title{\textbf{ECE 452 - Home work 2}}
    % \author{Amrutha Varshini Ramesh\\
    % Net ID : arames8\\
    % UIN : 663365516}
    % \maketitle
% This LaTeX was auto-generated from MATLAB code.
% To make changes, update the MATLAB code and republish this document.

% \documentclass{article}
% \usepackage{graphicx}


% \setlength{\parindent}{0pt}

% \begin{document}

    
\begin{verbatim}
function arames8_hw2()
\end{verbatim}
\begin{verbatim}
%1(a) Compute v_hat for v = [5, −1, −2]T
    v_hat =hat(5, -1, -2)
\end{verbatim}

        \color{lightgray} \begin{verbatim}
v_hat =

     0     2    -1
    -2     0    -5
     1     5     0

\end{verbatim} \color{black}
    \begin{verbatim}
%1(b) Compute R = e^(v_hat)
    R = expm(v_hat)
\end{verbatim}

        \color{lightgray} \begin{verbatim}
R =

    0.9487   -0.3147    0.0292
    0.2122    0.7027    0.6791
   -0.2343   -0.6381    0.7334

\end{verbatim} \color{black}
    \begin{verbatim}
%1(c)Give the geometric interpretation (axis, angle) of R.
    angle_R = acos((trace(R)-1)/2)
    axis_R = (1/(2*sin(angle_R)))*[(R(3,2)-R(2,3)) ;(R(1,3)-R(3,1)) ;(R(2,1)-R(1,2))]
\end{verbatim}

        \color{lightgray} \begin{verbatim}
angle_R =

    0.8060


axis_R =

   -0.9129
    0.1826
    0.3651

\end{verbatim} \color{black}
    \begin{verbatim}
%2. What vector do you get if you rotate the vector p = [5, 2, −4]T by 75 degrees around the 
axis described by the vector ω = [4, 1, −3]T?
    p=[5; 2; -4];
    w =[4; 1; -3];
    theta = 75;
    theta_radians = deg2rad(theta);

    %normalize w
    normalized_w = norm_vector(w)
    w1 = normalized_w(1); w2=normalized_w(2); w3=normalized_w(3);
    
    w_hat = hat(w1,w2,w3)

    %Find rotational matrix
     R_ppprime2 = expm(w_hat*theta_radians)

     %Find the vector coordinates after rotation
     p_prime = R_ppprime2* p

%      v0 = 1-cos(theta_radians); c0 = cos(theta_radians); s0 = sin(theta_radians);
%       R_ppprime1 = [(w1^2*v0)+ c0,    (w1*w2*v0)-(w3*s0), (w1*w3*v0)+(w2*s0);
%                  (w1*w2*v0)+(w3*s0),(w2^2*v0)+ c0,      (w2*w3*v0)-(w1*s0);
%                   (w1*w3*v0)-(w2*s0),(w2*w3*v0)+(w1*s0), (w3^2*v0)+ c0]
%      R_ppprime3 = eye(3) + w_hat*sin(theta_radians)+ transpose(w_hat)*w_hat*v0
%      p_prime = R_ppprime1* p
%      p_prime = R_ppprime3* p
\end{verbatim}

        \color{lightgray} \begin{verbatim}
normalized_w =

    0.7845
    0.1961
   -0.5883


w_hat =

         0    0.5883    0.1961
   -0.5883         0   -0.7845
   -0.1961    0.7845         0


R_ppprime2 =

    0.7149    0.6823   -0.1526
   -0.4543    0.2873   -0.8433
   -0.5315    0.6722    0.5154


p_prime =

    5.5499
    1.6763
   -3.3747

\end{verbatim} \color{black}
    \begin{verbatim}
 %3a. First rotate the object for 60 degrees around the x axis of the frame A. Next, rotate 
      the object for 45 degrees around the z axis of the rotated frame B.
    
      Rab = gen_rot('x',degtorad(60));
      Rbc = gen_rot('z',degtorad(45));
      Rb = Rbc*Rab
  \end{verbatim}
  
          \color{lightgray} \begin{verbatim}
  Rb =
  
      0.7071   -0.3536    0.6124
      0.7071    0.3536   -0.6124
           0    0.8660    0.5000
  
  \end{verbatim} \color{black}
    \begin{verbatim}
 %3b. First rotate the object for 60 degrees around the x axis of the 
 frame A. Next, rotate the object for 45 degrees around the z axis of 
 the frame A.
    
      Rac = Rab*Rbc
    \end{verbatim}
    
            \color{lightgray} \begin{verbatim}
    Rac =
    
        0.7071   -0.7071         0
        0.3536    0.3536   -0.8660
        0.6124    0.6124    0.5000

\end{verbatim} \color{black}
    \begin{verbatim}
%4b)Give the geometric interpretation (axis, angle) of the rotation described by R.

R_4 =[0.4619, -0.1189, -0.8790;
    -0.5615, -0.8063, -0.1860;
    -0.6866, 0.5794, -0.4392];

angle_R4 = acos((trace(R_4)-1)/2)
axis_R4 = (1/(2*sin(angle_R4)))*[(R_4(3,2)-R_4(2,3)) ;(R_4(1,3)-R_4(3,1)) 
;(R_4(2,1)-R_4(1,2))]
\end{verbatim}

        \color{lightgray} \begin{verbatim}
angle_R4 =

    2.6721


axis_R4 =

    0.8459
   -0.2126
   -0.4891

\end{verbatim} \color{black}
    \begin{verbatim}
%4a)Find the exponential coordinates of R
expo_coordinates = axis_R4*angle_R4
\end{verbatim}

        \color{lightgray} \begin{verbatim}
expo_coordinates =

    2.2603
   -0.5682
   -1.3070

\end{verbatim} \color{black}
    \begin{verbatim}
%5 A rotation matrix between frames A and B is given by: A point q is described in 
the frame A by a vector qa = [−4, 3, 5]T. What is the description of in the frame B.

qa =  [-4; 3; 5];
Rab = [0.6325, 0.2533, -0.7319;
      -0.7074, 0.5737, -0.4128;
       0.3154 0.7789 0.5421];
  qb = inv(Rab)* qa
\end{verbatim}

        \color{lightgray} \begin{verbatim}
qb =

   -3.0758
    4.6025
    4.4000

    ********************************
\end{verbatim} \color{black}
    \begin{verbatim}
    function hat = hat(a1,a2,a3)
     hat = [0, -a3, a2 ; a3, 0, -a1 ; -a2, a1, 0];
    end

    function norm_vector = norm_vector(M)
        if norm(M) ==1
            norm_vector =M;
        else
            norm_vector = normc(M);
        end
    end

    function rot_mat = gen_rot(axis,angle)
        if axis == 'x'
            rot_mat = [1,0,0; 0,cos(angle),-sin(angle);0,sin(angle),cos(angle)];
        elseif axis =='y'
            rot_mat = [cos(angle),0,sin(angle); 0,1,0; -sin(angle),0,cos(angle)];
        elseif axis == 'z'
            rot_mat = [cos(angle),-sin(angle),0; sin(angle),cos(angle),0; 0,0,1];
        end
    end
\end{verbatim}
\begin{verbatim}
end
\end{verbatim}



\end{document}
    
